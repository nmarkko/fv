paragraph{11 Condition coverage.}
\indent Condition coverage beleži koliko je puta svaka permutacija članova u Bulovom izrazu izazvala da kompletan izraz ima vrednost true ili false. Razmotrimo izraz (A && B) || C || D. Ovaj izraz je true u slučaju sledeća tri uslova:
\begin{enumerate}
\item A &&B
\item C
\item D
\end{enumerate}
a false je u slučaju ova dva uslova:
\begin{enumerate}
\item !A && !C && !D
\item !B && !C && !D
\end{enumerate}
\indent Stroža forma condition pokrivenosti - ekskluzivna condition pokrivenost - je ponekad na raspolaganju. Ona zahteva da svaki član bude kontrolišući član - da bude jedini razlog za to što izraz ima vrednost true ili false. Za gore navedeni primer izraz ima vrednost true za narednih pet ekskluzivnih uslova. Kontrolišući član je boldovan:
\begin{enumerate}
\item \textbf{A && B} && !C && !D
\item \textbf{A} && B && \textbf{C} && !D
\item A && \textbf{!B} && \textbf{C} && !D
\item \textbf{!A} && B && !C && \textbf{D}
\item A && \textbf{!B} && !C && \textbf{D}
\end{enumerate}
paragraph{12 FSM coverage.}
\indent Moderni alati za code coverage identifikuju i izdvajaju mašine konačnog stanja iz RTLa. Postoji više FSM metrika od interesa za verifikacione inženjere:
\begin{itemize}
\item[-] Osnovna je \textbf{state coverage}: koliko puta je mašina stanja ušla u svako od stanja?
\item[-] Druga je \textbf{arc coverage}: koliko puta je FSM prešao iz jednog stanja u njemu susedna stanja? Arc coverage bi trebao da bude reportovan (reported) za pod-izraze svake jednačine narednog stanja, kao što smo videli u condition pokrivenosti.
\item[-] Treća FSM metrika je \textbf{sequential arc coverage}, često nazvana \textbf{transition coverage}. Sequential arc coverage identifikuje sekvence posete stanja različitih trajanja i beleži broj obilazaka svake sekvence.
\end{itemize}
\indent Razmotrimo primer FSMa od pet stanja i devet grana (arcs) prikazanih na slici \ref{img-state-coverage}. Pored svake grane nalazi se njoj odgovarajuća jednačina narednog stanja. Moguć izveštaj state coverage-a za ovaj FSM dat je slici pored.\\
\indent Arc coverage za isti FSM dat je na slici \ref{img-arc-coverage}.\\
\indent Moguć izveštaj sequential arc coverage-a za dvo-grane prelaze počevši od stanja S1, za isti FSM, dat je na slici \ref{img-sequential-arc-coverage}.
paragraph{13 Neophodni koraci u procesu korišćenja strukturne pokrivenosti (Code Coverage). Navesti i ukratko objasniti svaki od njih.}
\indent Code coverage izveštava o tome koliko je uspešno RTL implementacija uređaja izupotrebljavana (exercised) iz perspektive ranije navedenih metrika. Pošto je RTL podložan promenama u ranim fazama dizajna, mi smo zainteresovani za izupotrebljavanost dizajna tek u kasnijim etapama dizajn ciklusa. Koraci code coverage procesa su:
\begin{itemize}
\item \textbf{instrumentacija}: Prvi korak u korišćenju strukturne (code) pokrivenosti je instrumentacija RTLa. Odabiramo module koda, hijerarhije ili instance koje ćemo posmatrati. Sledeće, odabiramo metriku koju ćemo beležiti (record). Količina instrumentovanog koda i broj merenih metrika će odrediti stepen degradacije simulacije. Moramo ograničiti ova dva faktora tako da postignemo ciljeve pokrivenosti. Poslednji korak instrumentacije zavisi od konkretnog alata koji koristimo:
\begin{itemize}
\item[-] Za započinjanje beleženja metrika neki alati ne zahtevaju dalje radnje pre simulacije.
\item[-] Drugi alati zahtevaju korak instrumentacije/kompilacije, gde u RTL ubacuju kod koji definiše brojače i inkrementira ih.
\end{itemize}
\item \textbf{beleženje metrike}: Drugi korak u primeni strukturne pokrivenosti je beleženje metrika. Ovo beleženje radi simulator tokom simulacije. Zabeleženi podaci moraju da se organizuju pre analize koja sledi. Svaka od zabeleženih metrika ima svoj prag ili hit count. Default vrednost obučno iznosi 1. U visoko rizičnim delovima RTLa, sa možda neobičnom količinom kompleksnosti, trebalo bu razmotriti povećanje praga zabeleženih metrika u ovim delovima. Ovo "over-samplovanje" će umanjiti rizik be
\item \textbf{analiza}: Treći korak u korišćenju strukturne pokrivenosti je analiziranje merenja. Irelevantni podaci moraju da se isključe iz analiza i fokus mora da bude na značenju zabeleženih metrika. Treba zapamtiti da strukturna pokrivenost ne može da otkrije potreban RTL, koji nedostaje. Ako RTL potreban za implementaciju dela funkcionalnosti uređaja nije napisan, to ne može da identifikuje strukturna pokrivenost, već samo funkcionalna. Kako bi se smanjila preopterećenost informacijama dobijenih od alata za pokrivenost koristi se data filtering.
\end{itemize}
\indent *Random useful fact: Structural coverage models have no ability to predict bugs; bug coverage model they are not.
paragraph{14 Funkcionalna pokrivenost (Functional Coverage). Osnovna ideja.}
\indent Svrha merenja funkcionalne pokrivenosti je merenje verifikacionog napretka iz perspektive funkcionalnih zahteva uređaja. Funkcionalni zahtevi za ulaze i izlaze uređaja - i njihove međusobne veze - specifikacijama uređaja (funkcionalna specifikacija i specifikacija dizajna). Zahtevi za ulaze diktiraju opseg podataka i vremenski opseg ulaznih stimulusa. Zahtevi za izlaze specificiraju potpuni skup data i temporalnih odgovora koji se posmatraju. Ulazno/izlazni zahtevi specificiraju sve stimulus/response permutacije koje se moraju posmatrati kako bi se zadovoljili black-box device zahtevi.\\
\indent Pošto ovi ulazni, izlazni i ulazno/izlazni zahtevi mogu u potpunosti da definišu ponašanje uređaja, prostor funkcionalne pokrivenosti koji oslikava ove zahteve se naziva model pokrivenosti (coverage model). Stepen u kom coverage model oslikava ove zahteve jeste njegova tačnost (fidelity). Fidelity modela određuje koliko blisko model definiše stvarne bihevijalne zahteve uređaja. Ovo je abstraction gap između modela pokrivenosti i uređaja.\\
\indent Za razliku od strukturne pokrivenosti, ne postoji automatizovan način za kreiranje modela funkcionalne pokrivenosti. Funkcionalna pokrivenost targetira semantičke aspekte test generisanja ili dizajn implementacije. Neophodno je odabrati koje funkcionalne oblasti dizajna treba testirati. Uvid u dizajn semantiku navodi ove izbore, i te uvide daju dizajner ili verifikacioni inženjeri.\\
\indent Značajna komponenta ovih uvida je znanje o kompleksnosti dizajna, za oblasti koje su sklone bugovima. Predictive bug coverage komponenta modela funkcionalne pokrivenosti potiče od iskustva inženjera. Ne postoji skup kompleksnih automatskih alata na raspolaganju za definisanje modela funkcionalne pokrivenosti. Alati većinom podržavaju implementaciju modela pokrivenosti i efikasno skupljanje podataka tokom simulacije (during runtime). Nakon što su podaci prikupljeni, alati pružaju GUI podršku za pregled rezultata.
paragraph{15 Koraci prilikom formiranja modela funkcionalne pokrivenosti. Navesti i ukratko objasniti.}
\indent Koraci prilikom formiranja modela funkcionalne pokrivenosti uključuju:
\begin{itemize}
\item opis semantike modela,
\item identifikovanje atributa (eventova) pokrivenosti i
\item specificiranje veze između ovih atributa koji karakterišu uređaj i vreme njihove korelacije. 
\end{itemize}
\indent Semantika modela je priča, opis onoga što se modeluje. Primer input coverage modela je: \textit{Dekoder instrukcija mora da dekodira svaki opkod, u svakom modu adresiranja sa svim permutacijama operand registara}. Primer output modela je: \textit{Posmatramo sekvencu paketa istog prioriteta, gde dužina sekvence varira od 1 do 255}.\\
\indent Kada je napisan opis semantike, drugi kroak u diyajniranju modela pokrivenosti je identifikacija atributa, tj. coverage eventova. Šta podrazumevamo pod atributom? Atribut specificira događaj u modelu ili testbenchu, koji je dovoljno značajan da bi ga verifikaciono okruženje zapazilo i zabeležilo njegovu pojavu..[...].
paragraph{16 Matrični model zavisnosti atributa funkcionalne pokrivenosti. Objasniti i navesti primer.}
\indent Matrični model razmatra svaki atribut kao dimenziju matrice, gde je broj dimenzija definisan brojem atributa. Vrednosti duž svake ose su vrednosti odgovarajućih atributa. Slika \ref{img-matrix-coverage-model} ilustruje dvodimenzionalni matrični model. Dva atributa su obeležena kao "Atribut A" i "Atribut B". Atribut A ima dvanaest vrednosti, od A0 do A11 dok atribut B ima osam vrednosti, od B0 do B7. Par atributa (An,Bm) definiše tačku u dvodimenzionalnom prostoru pokrivenosti. Ovaj matrični model definiše 96 tačaka.
%ubaci sliku
paragraph{17 Hijerarhijski model zavisnosti atributa funkcionalne pokrivenosti. Objasniti i navesti primer.}
\indent Hijerarhijski model ima strukturu invertovanog stabla sa korenom na vrhu. To je usmereni graf
paragraph{18 Hibridni model zavisnosti atributa funkcionalne pokrivenosti. Objasniti i navesti primer.}
paragraph{19 Navesti tri modela zavisnosti atributa funkcionalne pokrivenosti. Dobre i loše strane.}
paragraph{20 Semplovanje ulaznih atributa u slučaju funkcionalne pokrivenosti.}
paragraph{21 Semplovanje izlaznih atributa u slučaju funkcionalne pokrivenosti.}
paragraph{22 Semplovanje unutrašnjih atributa u slučaju funkcionalne pokrivenosti.}
